\section{Атаки по сторонним каналам} \label{sec:sca}

В данной главе рассматриваются несколько типов атак по сторонним каналам с
точки зрения их применимости к программам под Linux. В качестве критериев для
оценки применимости были выбраны сложность реализации атаки, допускаемые атакой
сценарии и требовуемое оборудование. В книге \cite{cren} и статье \cite{10years}
рассматриваются и сравниваются различные типы атак по сторонним каналам.

% Эта публикация очень помогла написанию текущей главы:
%  http://ru.bmstu.wiki/%D0%90%D1%82%D0%B0%D0%BA%D0%B8_%D0%BF%D0%BE_%D0%BF%D0%BE%D0%B1%D0%BE%D1%87%D0%BD%D1%8B%D0%BC_%D0%BA%D0%B0%D0%BD%D0%B0%D0%BB%D0%B0%D0%BC:_%D0%B4%D0%B5%D1%81%D1%8F%D1%82%D1%8C_%D0%BB%D0%B5%D1%82_%D0%BF%D0%BE%D1%81%D0%BB%D0%B5_%D0%BF%D1%83%D0%B1%D0%BB%D0%B8%D0%BA%D0%B0%D1%86%D0%B8%D0%B8_%D0%B8_%D0%B2%D0%BE%D0%B7%D0%B4%D0%B5%D0%B9%D1%81%D1%82%D0%B2%D0%B8%D0%B5_%D0%BD%D0%B0_%D0%BA%D1%80%D0%B8%D0%BF%D1%82%D0%BE%D0%B3%D1%80%D0%B0%D1%84%D0%B8%D1%87%D0%B5%D1%81%D0%BA%D0%B8%D0%B9_%D0%BC%D0%BE%D0%B4%D1%83%D0%BB%D1%8C_%D1%82%D0%B5%D1%81%D1%82%D0%B8%D1%80%D0%BE%D0%B2%D0%B0%D0%BD%D0%B8%D1%8F_%D0%B1%D0%B5%D0%B7%D0%BE%D0%BF%D0%B0%D1%81%D0%BD%D0%BE%D1%81%D1%82%D0%B8#.D0.90.D1.82.D0.B0.D0.BA.D0.B0_.D0.BF.D0.BE_.D0.B2.D1.80.D0.B5.D0.BC.D0.B5.D0.BD.D0.B8
%  Она же в оригинале: https://eprint.iacr.org/2005/388.pdf

% Эта глава местами сильно дублирует введение

\subsection{Атаки по времени}

Первый пример использования информации о физических процессах для проведения
атак на криптографические алгоритмы был продемонстрирован P. Kocher в 1996 году.
В своей публикации \cite{kocher} он показал, как наблюдаемая атакующим разница
во времени вычисления может быть использована для взлома некоторых реализаций
RSA и криптосистем, основанных на задаче дискретного логарифмирования.

Зависимость между входными данными криптосистемы и затраченным на вычисления
временем может быть вызвана: условными переходами в программе, зависящими от
входных данных, компиляторными оптимизациями, кэш-промахами при обращениях к
памяти \cite{bernstein}, промахами модуля прогнозирования ветвлений (англ.
branch prediction unit) \cite{bpa} \cite{sbpa}. Данные методы подробно описаны в
книге \cite{cren}.

Атаки по времени интересны тем, что они допускают сценарии, при которых
не требуется ни активного вмешательства в работу шифрующего устройства ни
физического доступа к нему для снятия изменений. Атакующий может находиться в
одной сети с уязвимым устройством и измерять время, которое уходит у него на
обработку запросов, которые поступают от легитимных клиентов или формируются
самим атакующим. Кроме того, он может удалённо слать такие запросы по сети
или обращаться с ними к шифрующему процессу, имея доступ непривелигированного
пользователя к шифрующей машине.

Основным препятствием для проведения успешной атаки по времени является шум,
который присутствует в измерениях, снимаемых атакующим. Этот шум вызван
множеством случайных процессов, которые влияют на измеряемое атакующим время.
Существует несколько статистических методов, которые применяются для анализа
подверженных шуму измерений при атаках по времени. Несмотря на присутствующий
в измерениях шум, эти методы позволяют получать надёжные результаты, требуя
большого числа измерений. Также существуют альтернативные не статистические
методы анализа измерений такие как нейронные сети. Аналитическая модель,
описывающая влияние шума на снимаемые атакующим измерения при атаках по времени,
описана в статье \cite{anmodel}.

\subsection{Атаки по энергопотреблению}

Мощность, потребляемая электронными устройствами, зависит от данных, которыми
они манипулируют, а также от выполняемых ими инструкций. Эта зависимость вызвана
конструкцией вентилей и триггеров, базовых элементов элементов цифровых схем.
При атаке по энергопотреблению атакующий измеряет потребляемую устройством
мощность при помощи осциллографа или самописца, которые подключаются к резистру,
вставленному в электрическую цепь последовательно с источником питания устройства.

Существует два типа атак по энергопотреблению, различающихся способами анализа
измерений, требованиями к точности измерений, а также сложностью реализации.
Атаки первого типа называются простым анализом энергопотребления (англ. SPA)
\cite{spa} \cite{spa1} \cite{kocher-dpa}. Они основаны на визуальном анализе
графика изменений энергопотребления с течением времени. Для отфильтровывания
шума, присутствующего в измерениях, могут применяться частотные фильтры и
усредняющие функции.

Вторым типом атак по энергопотреблению является дифференциальный анализ
энергопотребления (англ. DPA). Этот тип атак более продвинут и позволяет
вычислить промежуточные значения, использующиеся в уязвимом устройстве путём
статистического анализа многократно повторяемых измерений. Данный тип атак был
впервые представлен в работе \cite{kocher-dpa}.

Большинство работ, посвящённых данным типам атак, рассматривают в качестве
атакуемого устройства микроконтроллеры и смарт-карты, имеющие небольшой уровень
энергопотребления и не содержащие большого количества периферийных устройств,
вносящих в измерения допольнительный шум. Кроме того, измерения для атак по
энергопотреблению часто требуется снимать непосредственно с кабеля питания
вычисляющего устройства для уменьшения шума, создаваемого периферийными
устройствами расположеными на одной плате с ним.

\subsection{Атаки по электромагнитному излучению}

Первой публикацией, продемонстрировавшей возможность извлечения информации
из электромагритного излучения испускаемого устройством, была статья
\cite{van-eck}. В ней было показано, что электромагнитное излучение от монитора
компьютера может быть использовано для восстановления изображения отображаемого
монитором. Работы по получению информации о вычислениях, производимых
процессором или другой цифровой микросхемой, появились позже. В статьях
\cite{ema1} \cite{ema2} описаны атаки, которые требуют установки антенн на очень
близком расстоянии от атакуемой микросхемы, для чего требуется вскрытие корпуса
устройства.

Измеряемое при поведении данного типа атак электромагнитное излучение принято
разделять на два класса: прямое излучение и непроизвольное излучение. Прямое
излучение вызывается электрическими токами, которые протекают через элементы
микросхемы. Его зависимость от тока описывается законом Ампера-Максвелла,
описываемым уравнением \ref{eq:maxwell}.

\begin{equation}
\nabla \times \mathbf{B} = \mathbf{j} + \frac{\delta \mathbf{E}}{\delta t}
\end{equation} \label{eq:maxwell}

В уравнении \ref{eq:maxwell} используются следующие обозначения:

\begin{description} 
\item[$\mathbf{B}$] -- вектор магнитной индукции в теслах;
\item[$\mathbf{E}$] -- напряжённость электрического поля в вольтах на метр $t$ -- время в секундах;
\item[$\mathbf{j}$] -- плостность тока в амперах на квадратный метр.
\end{description}

Работы \cite{ema1} \cite{ema2} были посвящены атакам именно на такой тип
излучения. В сложных схемах выделение прямого излучения очень затруднено шумом,
создаваемым другими сигналами. Именно это обстоятельство и требует располагать
антенну для измерений очень близко к интересующей части микросхемы.

Непроизвольное излучение вызывается многчисленными электрическими и магнитными
взаимодействиями между отдельными компонентами схемы, которые размещаются очень
плотно на современных микросхемах, в зависимости от их формы и положения.
Большая часть таких излучений игнорируется разработчиками микросхем, так как не
оказывает существенного влияния на функционал. Тем не менее, такие излучения
могут быть источником угрозы конфеденциальности обрабатываемой устройством
информации.

Непроизвольное излучение намного сильнее прямого и распространяется намного
дальше, что позволяет измерять его не прибегая к вмешательству в работу
устройства или вскрытию корпуса. Основными источниками непроизвольного
излучения являются тактовые сигналы и сигналы, используемые для взаимодействия
с внешними устройствами.

Оба описынных выше типа атак по электромагнитному излучению требуют наличия
высокоточного приёмника/демодулятора, который может быть настроен на различные
значения несущей частоты, а также способен производить демодуляцию измерений
для извлечения интересующего сигнала. Такое оборудование, как правило, является
достаточно дорогим. Это требование значительно затрудняет проведение атак по
электромагнитному излучению.

\subsection{Выбор наиболее переспективного типа атак}

Таким образом, наиболее переспективной с точки зрения применения к программе
валидации пароля под Linux является атака по времени. Другие типы атак требуют
дорогостоящее оборудование для проведения или вмешатетьство в работу устройства,
а также не предоставляют такой гибкости в выборе сценария атаки, какая доступна
атакам по времени.

\clearpage
