\section{Атаки по сторонним каналам}
\clearpage

\subsection{Атаки по времени}

Первый пример использования информации о физических процессах для проведения
атак на криптографические алгоритмы был продемонстрирован P. Kocher в 1996 году.
В своей публикации \cite{kocher} он показал, как наблюдаемая атакующим разница
во времени вычисления может быть использована для взлома некоторых реализаций
RSA и криптосистем, основанных на задаче дискретного логарифмирования.

Зависимость между входными данными криптосистемы и затраченным на вычисления
временем может быть вызвана: условными переходами в программе, зависящими от
входных данных, компиляторными оптимизациями, кэш-промахами при обращениях к
памяти, промахами модуля прогнозирования ветвлений (англ. branch prediction
unit). Данные методы подробно описаны в книге \cite{cren}.

Атаки по времени интересны тем, что они допускают сценарии, при которых
не требуется ни активного вмешательства в работу шифрующего устройства ни
физического доступа к нему для снятия изменений. Атакующий может находиться в
одной сети с шифрующим устройством и измерять время, которое уходит у него на
обработку запросов на шифрование, которые поступают от легитимных клиентов.
Кроме того, он может удалённо слать такие запросы по сети или обращаться с ними
к шифрующему процессу, имея доступ непривелигированного пользователя к шифрующей
машине.

Основным препятствием для проведения успешной атаки по времени является шум,
который присутствует в измерениях, снимаемых атакующим. Этот шум вызван
множеством случайных процессов, которые влияют на измеряемое атакующим время.
Существует несколько статистических методов, которые применяются для анализа
подверженных шуму измерений при атаках по времени. Несмотря на присутствующий
в измерениях шум, эти методы позволяют получать надёжные результаты, требуя
большого числа измерений. Также существуют альтернативные не статистические
методы анализа измерений такие как нейронные сети.

\clearpage

\subsection{Атаки по энергопотреблению}
\clearpage

\subsection{Атаки по электромагнитному излучению}
\clearpage

\subsection{Атаки на кэш}
\clearpage

\subsection{Атаки на branch prediction unit}
\clearpage

\subsection{Итог анализа}
\clearpage
