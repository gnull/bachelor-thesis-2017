\section{Атаки по сторонним каналам} \label{sec:sca}

В данной главе рассматриваются несколько типов атак по сторонним каналам с
точки зрения их применимости к программам под Linux. В качестве критериев для
оценки применимости были выбраны сложность реализации атаки, допускаемые атакой
сценарии и требовуемое оборудование. В книге \cite{cren} и статье \cite{10years}
рассматриваются и сравниваются различные типы атак по сторонним каналам.

% Эта публикация очень помогла написанию текущей главы:
%  http://ru.bmstu.wiki/%D0%90%D1%82%D0%B0%D0%BA%D0%B8_%D0%BF%D0%BE_%D0%BF%D0%BE%D0%B1%D0%BE%D1%87%D0%BD%D1%8B%D0%BC_%D0%BA%D0%B0%D0%BD%D0%B0%D0%BB%D0%B0%D0%BC:_%D0%B4%D0%B5%D1%81%D1%8F%D1%82%D1%8C_%D0%BB%D0%B5%D1%82_%D0%BF%D0%BE%D1%81%D0%BB%D0%B5_%D0%BF%D1%83%D0%B1%D0%BB%D0%B8%D0%BA%D0%B0%D1%86%D0%B8%D0%B8_%D0%B8_%D0%B2%D0%BE%D0%B7%D0%B4%D0%B5%D0%B9%D1%81%D1%82%D0%B2%D0%B8%D0%B5_%D0%BD%D0%B0_%D0%BA%D1%80%D0%B8%D0%BF%D1%82%D0%BE%D0%B3%D1%80%D0%B0%D1%84%D0%B8%D1%87%D0%B5%D1%81%D0%BA%D0%B8%D0%B9_%D0%BC%D0%BE%D0%B4%D1%83%D0%BB%D1%8C_%D1%82%D0%B5%D1%81%D1%82%D0%B8%D1%80%D0%BE%D0%B2%D0%B0%D0%BD%D0%B8%D1%8F_%D0%B1%D0%B5%D0%B7%D0%BE%D0%BF%D0%B0%D1%81%D0%BD%D0%BE%D1%81%D1%82%D0%B8#.D0.90.D1.82.D0.B0.D0.BA.D0.B0_.D0.BF.D0.BE_.D0.B2.D1.80.D0.B5.D0.BC.D0.B5.D0.BD.D0.B8
%  Она же в оригинале: https://eprint.iacr.org/2005/388.pdf

\subsection{Атаки по времени}

Первый пример использования информации о физических процессах для проведения
атак на криптографические алгоритмы был продемонстрирован P. Kocher в 1996 году.
В своей публикации \cite{kocher} он показал, как наблюдаемая атакующим разница
во времени вычисления может быть использована для взлома некоторых реализаций
RSA и криптосистем, основанных на задаче дискретного логарифмирования.

Зависимость между входными данными криптосистемы и затраченным на вычисления
временем может быть вызвана: условными переходами в программе, зависящими от
входных данных, компиляторными оптимизациями, кэш-промахами при обращениях к
памяти \cite{bernstein}, промахами модуля прогнозирования ветвлений (англ.
branch prediction unit) \cite{bpa} \cite{sbpa}. Данные методы подробно описаны в
книге \cite{cren}.

Атаки по времени интересны тем, что они допускают сценарии, при которых
не требуется ни активного вмешательства в работу шифрующего устройства ни
физического доступа к нему для снятия изменений. Атакующий может находиться в
одной сети с уязвимым устройством и измерять время, которое уходит у него на
обработку запросов, которые поступают от легитимных клиентов или формируются
самим атакующим. Кроме того, он может удалённо слать такие запросы по сети
или обращаться с ними к шифрующему процессу, имея доступ непривелигированного
пользователя к шифрующей машине.

Основным препятствием для проведения успешной атаки по времени является шум,
который присутствует в измерениях, снимаемых атакующим. Этот шум вызван
множеством случайных процессов, которые влияют на измеряемое атакующим время.
Существует несколько статистических методов, которые применяются для анализа
подверженных шуму измерений при атаках по времени. Несмотря на присутствующий
в измерениях шум, эти методы позволяют получать надёжные результаты, требуя
большого числа измерений. Также существуют альтернативные не статистические
методы анализа измерений такие как нейронные сети. Аналитическая модель,
описывающая влияние шума на снимаемые атакующим измерения при атаках по времени,
описана в статье \cite{anmodel}.

\subsection{Атаки по энергопотреблению}

Мощность, потребляемая электронными устройствами, зависит от данных, которыми
они манипулируют, а также от выполняемых ими инструкций. Эта зависимость вызвана
конструкцией вентилей и триггеров, базовых элементов элементов цифровых схем.
При атаке по энергопотреблению атакующий измеряет потребляемую устройством
мощность при помощи осциллографа или самописца, которые подключаются к резистру,
вставленному в электрическую цепь последовательно с источником питания устройства.

Существует два типа атак по энергопотреблению, различающихся способами анализа
измерений, требованиями к точности измерений, а также сложностью реализации.
Атаки первого типа называются простым анализом энергопотребления (англ. SPA)
\cite{spa} \cite{spa1} \cite{kocher-dpa}. Они основаны на визуальном анализе
графика изменений энергопотребления с течением времени. Для отфильтровывания
шума, присутствующего в измерениях, могут применяться частотные фильтры и
усредняющие функции.

Вторым типом атак по энергопотреблению является дифференциальный анализ
энергопотребления (англ. DPA). Этот тип атак более продвинут и позволяет
вычислить промежуточные значения, использующиеся в уязвимом устройстве путём
статистического анализа многократно повторяемых измерений. Данный тип атак был
впервые представлен в работе \cite{kocher-dpa}.

Большинство работ, посвящённых данным типам атак, рассматривают в качестве
атакуемого устройства микроконтроллеры и смарт-карты, имеющие небольшой уровень
энергопотребления и не содержащие большого количества периферийных устройств,
вносящих в измерения допольнительный шум. Кроме того, измерения для атак по
энергопотреблению часто требуется снимать непосредственно с кабеля питания
вычисляющего устройства для уменьшения шума, создаваемого периферийными
устройствами расположеными на одной плате с ним.

\subsection{Атаки по электромагнитному излучению}
\clearpage

\subsection{Атаки на кэш}
\clearpage

\subsection{Атаки на branch prediction unit}
\clearpage

\subsection{Итог анализа}
\clearpage
