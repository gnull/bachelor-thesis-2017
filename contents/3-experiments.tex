\section{Эксперимент} \label{sec:experiment}

\subsection{Снятие измерений}

Запуск описанных в предыдущей секции программ осуществлялся на персональном компьютере с двухъядерным
процессором AMD E-300 под управлением операционной системы Linux с версией ядра
4.10.13. Более подробная информация о системе представлена в приложениях
\nameref{app:lscpu} и \nameref{app:uname}.

Для того, чтобы уменьшить шум, создаваемый множеством пользовательских процессов
и драйверов, которые обычно запущены на системе, измерения проводились без запуска
системы инициализации \texttt{systemd}. Для этого, при загрузке ядра, ему был
передан параметр \texttt{init=/bin/sh}, благодаря чему первым процессом, который
будет запущен ядром, будет оболочка \texttt{sh}. Из этой оболочки производился
запуск скрипта \nameref{app:collect}, вывод которого перенаправлялся в файл.

Использованная модель процессора поддерживает автоматическое изменение своей частоты
работы "на лету". Для избежания возможного влияния этой функции на точность измерений,
она была выключена вручную при помощи утилиты \texttt{cpupower},
а частота работы процессора установлена в неизменяемое значение 780 МГц.

Было снято $10^6$ измерений времени рельаной обработки запроса, которая была обозначена в
уравнении~\ref{eq:noise} как $X$, и столько же измерений времени, получаемого атакующим,
которе было обозначено в уравнении~\ref{eq:noise} как $Z$. Это было сделано при помощи
команд, представленных на листинге~\ref{src:collcalls}. При каждом измерении цикл
вызовов функции сравнения строк в программе \nameref{app:vuln} выполнял 10 итераций.

\begin{lstlisting}[caption=Команды снятия измерений, label=src:collcalls]
$ ./collect.pl ./vulnerable 1000000 10 5 > int-to-5-init.csv
$ ./collect.pl './timestamp ./vulnerable' 1000000 10 5 > ext-to-5-init.csv
\end{lstlisting}

В результате выполнения команд из листинга~\ref{src:collcalls} были получены два файла
\texttt{int-to-5-init.csv} и \texttt{ext-to-5-init.csv}. Каждый из которых содержит CSV-таблицу
из трёх столбцов:

\begin{description}
\item [\texttt{iterations}] -- число итераций цикла, представленного на
  листинге~\ref{src:vulnsnip}, совершённых при каждом вызове программы \nameref{app:vuln}
\item [\texttt{cycles}] -- измеренное время в циклах процессора
\item [\texttt{length}] -- длина общего префикса переданной программе \nameref{app:vuln} строки и
  внутренней строки, которая хранится в самой программе
\end{description}

\subsection{Анализ измерений}

В качестве инструмента для анализа снятых измерений было решено выбрать язык программирования
R \cite{r}, так как это очень мощный инструмент для статистического анализа и на нём реализованы
все требуемые для данной работы статистические алгоритмы.

\subsubsection{Оценка плотности распределения}

В данной секции представлены графики функций ядерной оценки плотности
распределения (kernel density estimate) для внутренних и внешних задержек.
Они позволяют визуально оценить форму расрпеделений полученных измерений, а так же
видуально сравнить измерения между собой.

\begin{figure}
    \centering
    \includegraphics[width=\textwidth]{data/sca-playground.wiki/strcmp_Timing_Attack/attempt-2/slides/int-to-5-init-non-outliers-density.png}
    \caption{Ядерная оценка плотности распределения внутренней задержки}
\end{figure} \label{fig:kde-in}

На графике \ref{fig:kde-in} легко заметить, что распределения задержек сравнения
строк, имеющих общие префиксы различной длины, схожи по форме, но различаются
средним значением, которое возрастает вместе с ростом длины общего префикса.

\begin{figure}
    \centering
    \includegraphics[width=\textwidth]{data/sca-playground.wiki/strcmp_Timing_Attack/attempt-2/slides/ext-to-5-init-density.png}
    \caption{Ядерная оценка плотности распределения внешней задержки}
\end{figure} \label{fig:kde-out}

На графике \ref{fig:kde-out} видно, что плотности распределения для всех длин
общего префикса настолько схожи, что сливаются в одну линию, и на первый взгляд
не удаётся рассмотреть на нём каких-то различий между ними.

\subsubsection{Точечные оценки}

Для сравнения того, насколько сильно шум, присутствующий во внешних измерениях,
повлиял на среднервадратичное отклонение задержки, были построены несмещённые
точечные оценки среднеквадратичного отклонения, которые представлены в следующих
талицах.

% Table created by stargazer v.5.2 by Marek Hlavac, Harvard University. E-mail: hlavac at fas.harvard.edu
% Date and time: Tue, May 23, 2017 - 08:11:20 AM
\begin{table}[!htbp] \centering 
\caption{Точечные оценки среднеквадратического отклонения внутренних измерений ($s_{in}$)} 
\label{pe_s_in} 
\begin{tabular}{@{\extracolsep{5pt}} ccc} 
\\[-1.8ex]\hline 
\hline \\[-1.8ex] 
Длина & $s_{in}$ \\ 
\hline \\[-1.8ex] 
$0$ & $51.764$ \\ 
$1$ & $49.222$ \\ 
$2$ & $50.634$ \\ 
$3$ & $50.772$ \\ 
$4$ & $52.566$ \\ 
\hline \\[-1.8ex] 
\end{tabular} 
\end{table} 

% Table created by stargazer v.5.2 by Marek Hlavac, Harvard University. E-mail: hlavac at fas.harvard.edu
% Date and time: Tue, May 23, 2017 - 08:29:22 AM
\begin{table}[!htbp] \centering 
\caption{Точечные оценки среднеквадратического отклонения внешних измерений ($s_{out}$)} 
\label{pe_s_out}
\begin{tabular}{@{\extracolsep{5pt}} ccc} 
\\[-1.8ex]\hline 
\hline \\[-1.8ex] 
Длина & $\s_{out}$ \\ 
\hline \\[-1.8ex] 
$0$ & $51,045.670$ \\ 
$1$ & $51,055.780$ \\ 
$2$ & $51,097.950$ \\ 
$3$ & $51,228.100$ \\ 
$4$ & $51,135.910$ \\ 
\hline \\[-1.8ex] 
\end{tabular} 
\end{table} 

Видно, что для внешних измерений точечные оценки превышают оценки внутренних
измерений примерно в $10^3$ раз.

% Table created by stargazer v.5.2 by Marek Hlavac, Harvard University. E-mail: hlavac at fas.harvard.edu
% Date and time: Tue, May 23, 2017 - 08:03:32 AM
\begin{table}[!htbp] \centering 
  \caption{Доверительные интервалы мат. ожидания} 
  \label{} 
\begin{tabular}{@{\extracolsep{5pt}} cccc} 
\\[-1.8ex]\hline 
\hline \\[-1.8ex] 
 & length & left & right \\ 
\hline \\[-1.8ex] 
1 & $0$ & $879.413$ & $879.867$ \\ 
2 & $1$ & $892.966$ & $893.398$ \\ 
3 & $2$ & $927.871$ & $928.315$ \\ 
4 & $3$ & $958.985$ & $959.430$ \\ 
5 & $4$ & $1,009.111$ & $1,009.571$ \\ 
\hline \\[-1.8ex] 
\end{tabular} 
\end{table} 

% Table created by stargazer v.5.2 by Marek Hlavac, Harvard University. E-mail: hlavac at fas.harvard.edu
% Date and time: Tue, May 23, 2017 - 09:01:52 AM
\begin{table}[!htbp] \centering 
  \caption{Линейная модель для внутренних измерений} 
  \label{} 
\begin{tabular}{@{\extracolsep{5pt}}lc} 
\\[-1.8ex]\hline 
\hline \\[-1.8ex] 
 & \multicolumn{1}{c}{\textit{Dependent variable:}} \\ 
\cline{2-2} 
\\[-1.8ex] & cycles \\ 
\hline \\[-1.8ex] 
 length & 32.556$^{***}$ (32.461, 32.650) \\ 
  Constant & 868.782$^{***}$ (868.551, 869.013) \\ 
 \hline \\[-1.8ex] 
Adjusted R$^{2}$ & 0.442 \\ 
Residual Std. Error & 51.739 (df = 999702) \\ 
F Statistic & 791,637.300$^{***}$ (df = 1; 999702) \\ 
\hline 
\hline
\textit{Note:}  & \multicolumn{1}{r}{$^{*}$p$<$0.1; $^{**}$p$<$0.05; $^{***}$p$<$0.01} \\ 
\end{tabular} 
\end{table} 

% Table created by stargazer v.5.2 by Marek Hlavac, Harvard University. E-mail: hlavac at fas.harvard.edu
% Date and time: Tue, May 23, 2017 - 09:05:06 AM
\begin{table}[!htbp] \centering 
  \caption{Линейная модель для внешних измерений} 
  \label{} 
\begin{tabular}{@{\extracolsep{5pt}}lc} 
\\[-1.8ex]\hline 
\hline \\[-1.8ex] 
 & \multicolumn{1}{c}{\textit{Dependent variable:}} \\ 
\cline{2-2} 
\\[-1.8ex] & cycles \\ 
\hline \\[-1.8ex] 
 length & 78.012$^{**}$ ($-$15.081, 171.105) \\ 
  Constant & 1.2e+6$^{***}$ (1,222,255, 1,222,711) \\ 
 \hline \\[-1.8ex] 
Adjusted R$^{2}$ & 0.00000 \\ 
Residual Std. Error & 51,112.740 (df = 999998) \\ 
F Statistic & 4.659$^{**}$ (df = 1; 999998) \\ 
\hline 
\hline \\[-1.8ex] 
\textit{Note:}  & \multicolumn{1}{r}{$^{*}$p$<$0.1; $^{**}$p$<$0.05; $^{***}$p$<$0.01} \\ 
\end{tabular} 
\end{table} 

\clearpage
