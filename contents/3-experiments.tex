\section{Эксперимент}

\subsection{Используемые инструменты}

Для исследований того, насколько велик шум, который влияет на измеряемое атакующим значение,
описанное в уравнении \ref{eq:noise}, была написана программа \ref{app:vuln}, которая сравнивает
передаваемую ей в качестве аргумента строку с внутренней строкой, состоящей из большого числа
символов \texttt{'a'}. Программа повторяет вызов функции сравнения в цикле заданное
пользователем число раз. Также эта программа измеряет время, которое было потрачено ею
на выполнение всего цикла сравнений и сообщает его пользователю. Это время будет являться
реальным времение обработки запроса, которо было обозначено как $X$ в уравнении \ref{eq:noise}.

На листинге \ref{src:vulnsnip} показана основная часть программы \ref{app:vuln}:

\begin{lstlisting}[caption=Отрывок программы \texttt{vulnerable.c}, label=src:vulnsnip]
int (*my_strcmp)(const char *s1, const char *s2) = strcmp;

char reference[] = "aaa...";

int main(int argc, char **argv)
{
	uint64_t begin, end;

	/* ... */

	begin = timestamp();
	for (int i = 0; i < n; ++i)
		my_strcmp(argv[1], reference);
	end = timestamp();

	printf("%lld\n", end - begin);
}
\end{lstlisting}

Для сравнения строк используется библиотечная функция \texttt{strcmp}. Вызовы к ней совершаются
через глобальный указатель \texttt{my\_strcmp}. Это было сделано для того, чтобы предотвратить оптимизации
компилятором вызовов к библиотечной функции \texttt{strcmp}, которые тот мог произвести так как программа
была собрана с флагом \texttt{-O3} для получения наиболее оптимального кода.

Время измерялось при помощи функции \texttt{timestamp}, представленной на листинге
\ref{src:timestamp}.

\begin{lstlisting}[caption=Функция \texttt{timestamp}, label=src:timestamp]
static inline uint64_t timestamp(void)
{
	uint32_t bottom;
	uint32_t top;
	asm volatile (
			"CPUID\n\t"
			"RDTSC\n\t"
			"mov %%edx, %0\n\t"
			"mov %%eax, %1\n\t": "=r" (top), "=r" (bottom)
#if __x86_64__
			:: "%rax", "%rbx", "%rcx", "%rdx");
#elif __i386__
			:: "%eax", "%ebx", "%ecx", "%edx");
#endif
	return (((uint64_t) top) << 32) | bottom;
}
\end{lstlisting}

Эта функция получает текущее время, обращаясь к счётчику тактов процессора при
помощи ассемблерной инструкции \texttt{rdtsc}, доступной на процессорах Intel x86.
Этот способ рекомендуется разработчиками Intel \cite{rdtsc} как наиболее точный для измерения
времени. Полученное при помощи этой инструкции значение представляет собой число
число тактов, прошедших с момента старта процессора, и может быть переведено в
секунды делением на тактовую частоту процессора. Так же, эта функция выполняет инструкцию
\texttt{cpuid} для того, чтобы предотвратить внеочердное выполнение
(англ. Out-of-order execution) процессором инструкций, которые следуют за \texttt{rdtsc}.

Для измерения времени, которое в соответствии c описанной в главе \ref{ch:timing}
моделью, соответсвовало бы времени, которое способен измерить атакующий, была
была написана программа \ref{app:timestamp}, эта программа вызывает программу
\ref{app:vuln} и дожидается её завершения при помощи последовательности системных
вызовов \texttt{fork}, \texttt{execve} и \texttt{wait}. При этом, она передаёт
программе \ref{app:vuln} указанные пользователем аргументы и измеряет время, которое
прошло между моментом, когда она создала новый процесс при помощи \texttt{fork},
и моментом, когда она узнала о его завершении через системный вызов \texttt{wait}.
Как и в программе \ref{app:vuln}, время измеряется при помощи функции \texttt{timetamp},
представленной на листинге \ref{src:timestamp}.

Для снятия большого числа измерений времени программами \ref{app:vuln} и
\ref{app:timestamp}, был написал Perl-скрипт \nameref{app:collect}. Основная логика его
работы преставлена на листинге \ref{src:collectsnip};

\begin{lstlisting}[caption=Фрагмент скрипта \texttt{collect.pl}, label=src:collectsnip]
my ($cmd, $reps, $its, $maxlen) = @ARGV;

print "iterations,length,cycles\n";

for (my $i = 0; $i < $reps; ++$i) {
	print STDERR "\r         \r$i / $reps" if ($i \% 100 == 0);

	my $len = int(rand $maxlen);
	my $str = "a" x $len . "b" x ($maxlen - $len);
	my $cycles = `$cmd $str $its`;
	chomp($cycles);
	print "$its,$len,$cycles\n"
}
\end{lstlisting}

Этот скрипт измеряет время выполнения цикла вызовов \texttt{strcmp} либо при помощи
программы \ref{app:vuln}, либо при помощи \ref{app:timestamp}, указанное пользователем
число раз и выводит на поток стандартного вывода таблицу измерений в формате CSV.
Для каждого измерения скрипт генерирует строку, которая совпадает с в внутренней строкой 
\nameref{app:vuln} на некоторое количество символов, которое выбирается при помощи генератора
псевдослучайных чисел для каждого запуска.

\subsection{Снятие измерений}

Запуск описанных в предыдущей секции программ осуществлялся на персональном компьютере с двухъядерным
процессором AMD E-300 под управлением операционной системы Linux с версией ядра
4.10.13. Более подробная информация о системе представлена в приложениях
\nameref{app:lscpu} и \nameref{app:uname}.

Для того, чтобы уменьшить шум, создаваемый множеством пользовательских процессов
и драйверов, которые обычно запущены на системе, измерения проводились без запуска
системы инициализации \texttt{systemd}. Для этого, при загрузке ядра, ему был
передан параметр \texttt{init=/bin/sh}, благодаря чему первым процессом, который
будет запущен ядром, будет оболочка \texttt{sh}. Из этой оболочки производился
запуск скрипта \nameref{app:collect}, вывод которого перенаправлялся в файл.

Использованная модель процессора поддерживает автоматическое изменение своей частоты
работы "на лету". Для избежания возможного влияния этой функции на точность измерений,
она была выключена вручную при помощи утилиты \texttt{cpupower},
а частота работы процессора установлена в неизменяемое значение 780 МГц.

Было снято $10^6$ измерений времени рельаной обработки запроса, которая была обозначена в
уравнении~\ref{eq:noise} как $X$, и столько же измерений времени, получаемого атакующим,
которе было обозначено в уравнении~\ref{eq:noise} как $Z$. Это было сделано при помощи
команд, представленных на листинге~\ref{src:collcalls}.

\begin{lstlisting}[caption=Команды снятия измерений, label=src:collcalls]
$ ./collect.pl ./vulnerable 1000000 10 5 > int-to-5-init.csv
$ ./collect.pl './timestamp ./vulnerable' 1000000 10 5 > ext-to-5-init.csv
\end{lstlisting}

В результате выполнения команд из листинга~\ref{src:collcalls} были получены два файла
\texttt{int-to-5-init.csv} и \texttt{ext-to-5-init.csv}. Каждый из которых содержит CSV-таблицу
из трёх столбцов:

\begin{description}
\item [\texttt{iterations}] число итераций цикла, представленного на
  листинге~\ref{src:vulnsnip}, совершённых при каждом вызове программы \nameref{app:vuln}
\item [\texttt{cycles}] измеренное время в циклах процессора
\item [\texttt{length}] длина общего префикса переданной программе \nameref{app:vuln} строки и
  внутренней строки, которая хранится в самой программе
\end{description}

\begin{figure}[H]
    \includegraphics[width=0.5\linewidth]{ext-to-5-init.png}
    \includegraphics[width=0.5\linewidth]{int-to-5-init.png} \label{length_entropy}
    \caption{Пара случайных картинок}
\end{figure}

\begin{figure}[H]
    \includegraphics[width=0.5\linewidth]{int-to-5-init-outliers-density.png}
    \includegraphics[width=0.5\linewidth]{int-to-5-init-non-outliers-density.png} \label{length_entropy}
    \caption{Пара случайных SVG-картинок}
\end{figure}

\clearpage

\section{Анализ результатов}

\clearpage
