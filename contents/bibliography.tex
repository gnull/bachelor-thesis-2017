\begingroup
\renewcommand{\section}[2]{\anonsection{Библиографический список}}
\begin{thebibliography}{00}

% This item is left here as an example. TODO: remove it
\bibitem{gonchar}
    Н. О. Гончаров.
    Современные угрозы ботсетей. [Электронный ресурс] // Молодежный научно-технический вестник №10. - октябрь 2014.
    URL: http://sntbul.bmstu.ru/file/out/734781.
    Режим доступа: свободный, дата обращения: 20.04.2016.

\bibitem{stargazer}
  Hlavac M. stargazer: LaTeX code and ASCII text for well-formatted regression and summary statistics tables //URL: \url{http://CRAN.R-project.org/package=stargazer}. – 2013.

\bibitem{r}
  Team R. C. R language definition //Vienna, Austria: R foundation for statistical computing. – 2000.

\bibitem{kocher}
  Kocher P. Timing attacks on implementations of Diffie-Hellman, RSA, DSS, and other systems //Advances in Cryptology—CRYPTO’96. – Springer Berlin/Heidelberg, 1996. – С. 104-113.

\bibitem{van-eck}
I Van Eck W. Electromagnetic radiation from video display units: An eavesdropping risk? //Computers \& Security. – 1985. – Т. 4. – №. 4. – С. 269-286.

\bibitem{asonov}
I Asonov D., Agrawal R. Keyboard acoustic emanations //Security and Privacy, 2004. Proceedings. 2004 IEEE Symposium on. – IEEE, 2004. – С. 3-11.

\bibitem{zhuang}
  Zhuang L., Zhou F., Tygar J. D. Keyboard acoustic emanations revisited //ACM Transactions on Information and System Security (TISSEC). – 2009. – Т. 13. – №. 1. – С. 3.

\bibitem{kuhn}
  Kuhn M. G. Electromagnetic eavesdropping risks of flat-panel displays //International Workshop on Privacy Enhancing Technologies. – Springer Berlin Heidelberg, 2004. – С. 88-107.

\bibitem{bernstein}
  Bernstein D. J. Cache-timing attacks on AES. – 2005.

\bibitem{remote-aes}
  Acıiçmez O., Schindler W., Koç Ç. K. Cache based remote timing attack on the AES //Cryptographers’ Track at the RSA Conference. – Springer Berlin Heidelberg, 2007. – С. 271-286.

\bibitem{rdtsc}
  Paoloni G. How to benchmark code execution times on Intel IA-32 and IA-64 instruction set architectures //Intel Corporation, September. – 2010. – Т. 123.

\bibitem{sbpa}
  Aciiçmez O., Koç Ç. K., Seifert J. P. On the power of simple branch prediction analysis //Proceedings of the 2nd ACM symposium on Information, computer and communications security. – ACM, 2007. – С. 312-320.

\bibitem{bpa}
  Acıiçmez O., Koç Ç. K., Seifert J. P. Predicting secret keys via branch prediction //Cryptographers’ Track at the RSA Conference. – Springer Berlin Heidelberg, 2007. – С. 225-242.

\bibitem{cren}
  Ko{\c{c}}, \c{C}.K. Cryptographic Engineering: Springer, 2008. 527 с

\bibitem{anmodel}
  Tiri K. et al. An analytical model for time-driven cache attacks //International Workshop on Fast Software Encryption. – Springer Berlin Heidelberg, 2007. – С. 399-413.

\bibitem{hands}
  Genkin D., Pipman I., Tromer E. Get your hands off my laptop: Physical side-channel key-extraction attacks on PCs //Journal of Cryptographic Engineering. – 2015. – Т. 5. – №. 2. – С. 95-112. 

\bibitem{em}
  Gandolfi K., Mourtel C., Olivier F. Electromagnetic analysis: Concrete results //International Workshop on Cryptographic Hardware and Embedded Systems. – Springer Berlin Heidelberg, 2001. – С. 251-261.

\bibitem{ssh}
  Song D. X., Wagner D., Tian X. Timing Analysis of Keystrokes and Timing Attacks on SSH //USENIX Security Symposium. – 2001. – Т. 2001. 

\end{thebibliography}
\endgroup

\clearpage
