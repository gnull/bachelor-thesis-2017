\begingroup
\renewcommand{\section}[2]{\anonsection{Список используемых источников}}
\begin{thebibliography}{00}

\bibitem{sbpa}
  Aciiçmez O., Koç Ç. K., Seifert J. P. On the power of simple branch prediction analysis //Proceedings of the 2nd ACM symposium on Information, computer and communications security. – ACM, 2007. – С. 312-320.

\bibitem{bpa}
  Acıiçmez O., Koç Ç. K., Seifert J. P. Predicting secret keys via branch prediction //Cryptographers’ Track at the RSA Conference. – Springer Berlin Heidelberg, 2007. – С. 225-242.

\bibitem{remote-aes}
  Acıiçmez O., Schindler W., Koç Ç. K. Cache based remote timing attack on the AES //Cryptographers’ Track at the RSA Conference. – Springer Berlin Heidelberg, 2007. – С. 271-286.

\bibitem{asonov}
  Asonov D., Agrawal R. Keyboard acoustic emanations //Security and Privacy, 2004. Proceedings. 2004 IEEE Symposium on. – IEEE, 2004. – С. 3-11.

\bibitem{bernstein}
  Bernstein D. J. Cache-timing attacks on AES – D. J. Bernstein's home page [Электронный ресурс] //URL: \url{https://cr.yp.to/antiforgery/cachetiming-20050414.pdf}, режим доступа: свободный, дата обращения 05.03.17.

\bibitem{ema2}
  Gandolfi K., Mourtel C., Olivier F. Electromagnetic analysis: Concrete results //International Workshop on Cryptographic Hardware and Embedded Systems. – Springer Berlin Heidelberg, 2001. – С. 251-261.

\bibitem{em}
  Gandolfi K., Mourtel C., Olivier F. Electromagnetic analysis: Concrete results //International Workshop on Cryptographic Hardware and Embedded Systems. – Springer Berlin Heidelberg, 2001. – С. 251-261.

\bibitem{hands}
  Genkin D., Pipman I., Tromer E. Get your hands off my laptop: Physical side-channel key-extraction attacks on PCs //Journal of Cryptographic Engineering. – 2015. – Т. 5. – №. 2. – С. 95-112. 

\bibitem{stargazer}
  Hlavac M. stargazer: LaTeX code and ASCII text for well-formatted regression and summary statistics tables -- The Comprehensive R Archive Network [Электронный ресурс] //URL: \url{http://CRAN.R-project.org/package=stargazer}, режим доступа: свободный, дата обращения 27.05.17.

\bibitem{kocher}
  Kocher P. Timing attacks on implementations of Diffie-Hellman, RSA, DSS, and other systems //Advances in Cryptology—CRYPTO’96. – Springer Berlin/Heidelberg, 1996. – С. 104-113.

\bibitem{kocher-dpa}
  Kocher P., Jaffe J., Jun B. Differential power analysis //Advances in cryptology—CRYPTO’99. – Springer Berlin/Heidelberg, 1999. – С. 789-789.

\bibitem{cren}
  Ko{\c{c}}, \c{C}.K. Cryptographic Engineering -- New York City: Springer, 2008. 527 с

\bibitem{kerrisk}
  Kerrisk M. The Linux programming interface. – No Starch Press, 2010.

\bibitem{kuhn}
  Kuhn M. G. Electromagnetic eavesdropping risks of flat-panel displays //International Workshop on Privacy Enhancing Technologies. – Springer Berlin Heidelberg, 2004. – С. 88-107.

\bibitem{spa}
  Mangard S. A simple power-analysis (SPA) attack on implementations of the AES key expansion //International Conference on Information Security and Cryptology. – Springer Berlin Heidelberg, 2002. – С. 343-358.

\bibitem{confint}
  Neyman J. Outline of a theory of statistical estimation based on the classical theory of probability //Philosophical Transactions of the Royal Society of London. Series A, Mathematical and Physical Sciences. – 1937. – Т. 236. – №. 767. – С. 333-380.

\bibitem{spa1}
  Oswald E. Enhancing simple power-analysis attacks on elliptic curve cryptosystems //International Workshop on Cryptographic Hardware and Embedded Systems. – Springer Berlin Heidelberg, 2002. – С. 82-97.

\bibitem{rdtsc}
  Paoloni G. How to benchmark code execution times on Intel IA-32 and IA-64 instruction set architectures //Intel Corporation, September. – 2010. – Т. 123.

\bibitem{kde}
  Parzen E. On estimation of a probability density function and mode //The annals of mathematical statistics. – 1962. – Т. 33. – №. 3. – С. 1065-1076.

\bibitem{ema1}
  Quisquater J. J., Samyde D. Electromagnetic analysis (ema): Measures and counter-measures for smart cards //Smart Card Programming and Security. – 2001. – С. 200-210.

\bibitem{ssh}
  Song D. X., Wagner D., Tian X. Timing Analysis of Keystrokes and Timing Attacks on SSH //USENIX Security Symposium. – 2001. – Т. 2001. 

\bibitem{r}
  Team R. C. R language definition //Vienna, Austria: R foundation for statistical computing. – 2000.

\bibitem{anmodel}
  Tiri K. et al. An analytical model for time-driven cache attacks //International Workshop on Fast Software Encryption. – Springer Berlin Heidelberg, 2007. – С. 399-413.

\bibitem{van-eck}
  Van Eck W. Electromagnetic radiation from video display units: An eavesdropping risk? //Computers \& Security. – 1985. – Т. 4. – №. 4. – С. 269-286.

\bibitem{10years}
  Zhou Y. B., Feng D. G. Side-Channel Attacks: Ten Years After Its Publication and the Impacts on Cryptographic Module Security Testing //IACR Cryptology ePrint Archive. – 2005. – Т. 2005. – С. 388.

\bibitem{zhuang}
  Zhuang L., Zhou F., Tygar J. D. Keyboard acoustic emanations revisited //ACM Transactions on Information and System Security (TISSEC). – 2009. – Т. 13. – №. 1. – С. 3.

\end{thebibliography}
\endgroup

\clearpage
