\begingroup
\renewcommand{\section}[2]{\anonsection{Библиографический список}}
\begin{thebibliography}{00}

% This item is left here as an example. TODO: remove it
\bibitem{gonchar}
    Н. О. Гончаров.
    Современные угрозы ботсетей. [Электронный ресурс] // Молодежный научно-технический вестник №10. - октябрь 2014.
    URL: http://sntbul.bmstu.ru/file/out/734781.
    Режим доступа: свободный, дата обращения: 20.04.2016.

\bibitem{stargazer}
  Hlavac, Marek (2015). stargazer: Well-Formatted Regression and Summary Statistics T
  R package version 5.2. \url{http://CRAN.R-project.org/package=stargazer}

\bibitem{kocher}
  P. C. Kocher. Timing attacks on implementations of Diffie-Hellman, RSA,
  DSS, and other systems. In N. Koblitz, editor, Advances in Cryptology −
  CRYPTO ’96, LNCS, vol. 1109, pp. 104–113. Springer-Verlag, 1996.

\bibitem{bernstein}
  Bernstein, D. J. (2005). Cache-timing attacks on AES.

\bibitem{remote-aes}
  Onur Acıiçmez, Werner Schindler, Çetin K. Koç.
  Cache Based Remote Timing Attack on the AES //
  Topics in Cryptology - 2007.

\bibitem{rdtsc}
  Gabriele P. How to Benchmark Code Execution Times on Intel IA-32 and IA-64 Instruction Set Architectures //
  \url{https://www.intel.com/content/dam/www/public/us/en/documents/manuals/64-ia-32-architectures-optimization-manual.pdf}

\bibitem{cren}
  Çetin Kaya Koç.
  Cryptographic Engineering. //
  Springer US. - 5(10):9, 2013.

\bibitem{anmodel}
  Tiri, K., Acıiçmez, O., Neve, M., \& Andersen, F. (n.d.).
  An Analytical Model for Time-Driven Cache Attacks.
  Fast Software Encryption, 399–413. \url{https://doi.org/10.1007/978-3-540-74619-5_25}



\end{thebibliography}
\endgroup

\clearpage
