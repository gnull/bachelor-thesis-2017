\anonsection{Список сокращений и условных обозначений}

\begin{description}
\item[ЭЛТ-монитор] -- монитор на основе электронно-лучевой турбки.
\item[ЖК-монитор] -- монитор на основе жидкокристаллического дисплея.
\item[BPU] -- англ. branch prediction unit, модуль прогнозирования ветвлений.
\item[SPA] -- англ. simple power analysis, простой анализ энергопотреблениaя.
\item[DPA] -- англ. differential power analysis, дифференциальный анализ
  энергопотребления.
\item[EMA] -- англ. electromagnetic analysis, анализ электромагнитного
  излучения.
\item[SEMA] -- англ. simple electromagnetic analysis, простой анализ
  электромагнитного излучения.
\item[DEMA] -- англ. differential electromagnetic analysis, дифференциальный
  анализ электромагнитного излучения.
\item[KDE] -- англ. kernel density estimate, ядерная оценка плотности распределения.
\item[TCP] -- англ. transmission control protocol, протокол управления передачей данных.
\item[UDP] -- англ. user datagram protocol, протокол пользовательских датаграмм.
\item[СКО] -- среднеквадратическое отклонение.
\item[МО] -- математическое ожидание.
\end{description}

\clearpage

\anonsection{Словарь терминов}

\begin{description}
\item[Аутентификация] -- установление соответствия между тем, кто пытается осуществить доступ к ресурсу, и существующей учётной записью в системе.
\item[Сокет] -- программный интерфейс для обеспечения обмена данными между процессами, работающими на одной или нескольких ЭВМ.
\item[Терминал] -- устройство, используемое для взаимодействия пользователя с компьютером или компьютерной системой, локальной или удалённой.
\end{description}

\clearpage
