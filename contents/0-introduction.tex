% -*- root: main.tex -*-
\anonsection{Введение}

Классическая криптография оценивает стойкость криптосистемы, рассматривая
сценарии, в которых атакующий получает доступ к некоторым входным параметрам
и/или результатам работы криптографических алгоритмов. При этом, не
рассматриваются детали конкретной программной или аппаратной реализации
алгоритма, внимание уделяется только математическому описанию алгоритма.

На практике же математические описания криптографических алгоритмов реализуются
в виде программы или цифровой схемы. Эти реализации могут предоставлять
атакующему много дополнительной информации о работе алгоритма. Такой информацией
может быть электромагнитное излучение, исходящее от шифрующего устройства,
потрябляемая им мощность, издаваемые им звуки, затраченное на вычисления время и
много других параметров.

Цель данной работы состоит в исследовании того, насколько релизуемы на практике
атаки по времени в операционной системе Linux. Для достижения этой цели были
поставлены две задачи. Первая задача состоит в написании программы, которая
имела бы зависимость между входными данными и временем исполнения, а так же
снятии измерений в окружении, которое соответствовало бы реалистичному сценарию
для атаки по времени. Вторая задача состоит в анализе полученных измерений,
формировании выводов о статистических свойствах измеряемых величин и шума, а
также о применимости рассмотренных статистических методов анализа измерений.

\clearpage
