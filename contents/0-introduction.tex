% -*- root: main.tex -*-
\anonsection{Введение}

Классическая криптография оценивает стойкость криптосистемы, рассматривая
сценарии, в которых атакующий получает доступ к некоторым входным параметрам
и/или результатам работы криптографических алгоритмов. При этом, не
рассматриваются детали конкретной программной или аппаратной реализации
алгоритма, внимание уделяется только математическому описанию алгоритма.

На практике же математические описания криптографических алгоритмов реализуются
в виде программы или цифровой схемы. Эти реализации могут предоставлять
атакующему много дополнительной информации о работе алгоритма. Такой информацией
может быть электромагнитное излучение, исходящее от шифрующего устройства,
потрябляемая им мощность, издаваемые им звуки, затраченное на вычисления время и
много других параметров, получаемых по сторонним каналам. Одна из первых атак по
времени на криптографические алгоритмы была описана в статье \cite{kocher}. Эта
атака позволяет извлечь секретные ключи протокола обмена ключами Диффи-Хеллмана
или закрытые ключи криптосистемы RSA, измеряя время, затраченное на операции
возведения в степень по модулю. Другая атака по времени описана в статье
\cite{bernstein}. Она основывается на различии во времени обращения к памяти,
вызываемом кэш-промахами. Атаки, основанные на анализе электромагнитного
излучения или колебаниях потенциала на корпусе компьютера, описаны в работах
\cite{hands} \cite{em}. Подробный обзор атак по сторонним каналам на реализации
криптографических алгоритмов сделан в книге \cite{cren}.

Атаки по сторонним каналам применимы не только к реализациям криптографических
алгоритмов. К ним могут быть уязвимы и другие типы устройств. Одной из первых
работ, посвящённых возможности извлечения информации из электромагнитного
излучения, испускаемого устройством, была статья \cite{van-eck}, опубликованная
в 1985 году. В ней была продемострирована возможность перехвата сожеримого
экрана ЭЛТ-монитора путём обнаружения распространяемых им электромагнитных волн.
Позже, в 2004 году, похожая атака была проведена на ЖК-монитор \cite{kuhn}.
Существуют работы, демонстрирующие акустические атаки на клавиатуры
\cite{asonov} \cite{zhuang}. Также известна атака по времени на ввод пароля пароля
при авторизации по протоколу SSH \cite{ssh}.

Цель данной работы состоит в исследовании того, насколько релизуемы на практике
атаки по времени на программу верификации пароля, запущенную на персональном
компьютере под операционной системой Linux. Программа верификации пароля
принимает от пользователя одним аргументом символьную строку, которую она
сравнивает с секретной строкой -- паролем, и в зависимости от того, совпадают
ли эти строки, разрешает или запрещает пользователю в выполнение каких-то
действий. Сравнение строк производится программой посимвольно и прекращается
при нахождении первого несовпадения. Программа верификации пароля была выбрана
для исследования, так как она является достаточно простым примером программы,
потенциально уязвимой в атакам по сторонним каналам.

Для достижения указанной выше цели, были поставлены следующие задачи:
\begin{enumerate}
\item Провести анализ и выбрать наиболее перспективные типы атак по сторонним
  каналам с точки зрения их применимости к программам под Linux;
\item Разработать модель программы верификации пароля;
\item Проанализировать полученные результаты;
\end{enumerate}

В главе \ref{sec:sca} написано...

В главе \ref{sec:timing} написано...

В главе \ref{sec:experiment} написано...

В главе \ref{sec:summary} написано...

\clearpage
