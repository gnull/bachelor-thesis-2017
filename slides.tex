\documentclass[14pt]{beamer}
\usepackage[english, russian]{babel}
\usepackage[utf8x]{inputenc}
\usepackage{itmobeamer}
\usepackage[labelformat=empty]{caption}

\title[Атаки по сторонним каналам]{}
\author[]{Олейников Иван}
\institute[]{Университет ИТМО}
\date[]{Санкт-Петербург, 2017}

\begin{document}

\begin{darkbars}
    \begin{frame}[noheader,nologo,noframenumbering]
        \titlepage
    \end{frame}
\end{darkbars}

\begin{frame}{Цель работы}
    Исследование эффективности атак по сторонним каналам на программу валидации
    пароля в ОС Linux. \\[3mm]
    Задачи:
    \begin{itemize}
        \item Выбор наиболее перспективных типов атак по сторонним каналам к
              программам под ОС Linux.
        \item Анализ атак по сторонним каналам на применимость к программе валидации
              пароля.
        \item Разработка модели уязвимой программы валидации пароля.
        \item Анализ полученных результатов.
    \end{itemize}
\end{frame}

\begin{frame}[nologo]{Типы атак по сторонним каналам}
    \begin{itemize}
        \item Атаки на кэш (access-based)
        \item Атаки по времени
        \item Атаки по энергопотреблению
        \item Электромагнитные атаки
        \item Акустические атаки
        \item Атаки со внесением ошибок
        \item Row-hammer
    \end{itemize}
\end{frame}

\begin{frame}{Ограничения атак по сторонним каналам}
    \begin{itemize}
        \item Требуется доступ к системе
            \begin{itemize}
                \item Возможность запуска программ на системе: атаки на кэш, Row-hammer
                \item Физический доступ: атаки по энергопотреблению, электромагнитные, акустические, внемением ошибок
            \end{itemize}
        \item Требуется высокоточное и дорогостоящее оборудование
            \begin{itemize}
                \item Самописцы, осциллографы: атаки по энергопотреблению, элкетромагнитные
                \item Микрофоны: акустические атаки
                \item Лазеры: атаки внесением ошибок
            \end{itemize}
    \end{itemize}
\end{frame}

\begin{frame}{Атаки по времени}
    \begin{itemize}
        \item Могут проводиться удалённо
        \item Намного проще других типов атак
        \item Современные процессоры позволяют измерять время с точностью до наносекунд
        \item Многие сетевые карты позволяют узнать точное время отправки/приёма пакета
        \item Не требуют наличия учёной степени по электротехнике
    \end{itemize}
\end{frame}

\begin{frame}{Модель атаки по времени}
    \begin{figure}
        \centering
        \includegraphics[height=0.9\textheight]{images/timing-attack-msc.png}
    \end{figure}
\end{frame}

\begin{frame}[nologo]
    \begin{figure}
        \centering
        \includegraphics[width=\textwidth]{data/sca-playground.wiki/strcmp_Timing_Attack/attempt-2/slides/int-to-5-init-non-outliers-density.png}
        \caption{Ядерная оценка плотности распределения внутренней задержки}
    \end{figure}
\end{frame}

\begin{frame}[nologo]
    \begin{figure}
        \centering
        \includegraphics[width=\textwidth]{data/sca-playground.wiki/strcmp_Timing_Attack/attempt-2/slides/ext-to-5-init-density.png}
        \caption{Ядерная оценка плотности распределения внешней задержки}
    \end{figure}
\end{frame}

\begin{frame}[nologo]{Параметры внутренней задержки}
  % Table created by stargazer v.5.2 by Marek Hlavac, Harvard University. E-mail: hlavac at fas.harvard.edu
  % Date and time: Tue, May 23, 2017 - 08:11:20 AM
  \begin{table}[!htbp] \centering 
    \caption{Точечные оценки среднеквадратического отклонения ($\sigma$)} 
    \label{} 
  \begin{tabular}{@{\extracolsep{5pt}} ccc} 
  \\[-1.8ex]\hline 
  \hline \\[-1.8ex] 
   Длина & $\sigma$ \\ 
  \hline \\[-1.8ex] 
  $0$ & $51.764$ \\ 
  $1$ & $49.222$ \\ 
  $2$ & $50.634$ \\ 
  $3$ & $50.772$ \\ 
  $4$ & $52.566$ \\ 
  \hline \\[-1.8ex] 
  \end{tabular} 
  \end{table} 
\end{frame}

\begin{frame}[nologo]{Параметры внутренней задержки}
% Table created by stargazer v.5.2 by Marek Hlavac, Harvard University. E-mail: hlavac at fas.harvard.edu
% Date and time: Tue, May 23, 2017 - 08:03:32 AM
\begin{table}[!htbp] \centering 
  \caption{Доверительные интервалы мат. ожидания} 
  \label{} 
\begin{tabular}{@{\extracolsep{5pt}} cccc} 
\\[-1.8ex]\hline 
\hline \\[-1.8ex] 
 Длина & $\mu - ME$ & $\mu + ME$ \\ 
\hline \\[-1.8ex] 
 $0$ & $879.413$ & $879.867$ \\ 
 $1$ & $892.966$ & $893.398$ \\ 
 $2$ & $927.871$ & $928.315$ \\ 
 $3$ & $958.985$ & $959.430$ \\ 
 $4$ & $1,009.111$ & $1,009.571$ \\ 
\hline \\[-1.8ex] 
\end{tabular} 
\end{table} 

$ME$ -- предел погрешности для уровня надёжности $99\%$ \\
$\mu$ -- точечная оценка мат. ожидания

\end{frame}

\begin{frame}{Линейная регрессия с одной независимой переменной}
  \begin{equation}
  X = \beta_L \times L + \beta_0 + \varepsilon
  \end{equation}

  $X$ -- время \\
  $L$ -- длина строки \\
  $\varepsilon$ -- случайная ошибка \\
\end{frame}

\begin{frame}[nologo]
  % Table created by stargazer v.5.2 by Marek Hlavac, Harvard University. E-mail: hlavac at fas.harvard.edu
  % Date and time: Tue, May 23, 2017 - 09:01:52 AM
  \begin{table}[!htbp] \centering 
    \caption{Параметры линейной регрессии для внутренних измерений} 
    \label{} 
  \begin{tabular}{@{\extracolsep{5pt}}lc} 
  \\[-1.8ex]\hline 
  \hline \\[-1.8ex] 
   & \multicolumn{1}{c}{\textit{Зависимая переменная:}} \\ 
  \cline{2-2} 
  \\[-1.8ex] & X \\ 
  \hline \\[-1.8ex] 
    $\beta_L$ & 32.556$^{***}$ (32.461, 32.650) \\ 
    $\beta_0$ & 868.782$^{***}$ (868.551, 869.013) \\ 
   \hline \\[-1.8ex] 
  $SE(\beta_L)$ & 51.739 (df = 999702) \\ 
  \hline 
  \hline
  \textit{Примечание:}  & \multicolumn{1}{r}{$^{*}$p$<$0.1; $^{**}$p$<$0.05; $^{***}$p$<$0.01} \\ 
  \end{tabular} 
  \end{table}
  $SE(\beta_L)$ -- стандартная ошибка для $\beta_L$
  %% http://statistica.ru/theory/osnovy-lineynoy-regressii/
\end{frame}

\begin{frame}{Параметры внешней задержки}
  % Table created by stargazer v.5.2 by Marek Hlavac, Harvard University. E-mail: hlavac at fas.harvard.edu
  % Date and time: Tue, May 23, 2017 - 08:29:22 AM
  \begin{table}[!htbp] \centering 
    \caption{Точечные оценки среднеквадратического отклонения ($\sigma$)} 
    \label{} 
  \begin{tabular}{@{\extracolsep{5pt}} ccc} 
  \\[-1.8ex]\hline 
  \hline \\[-1.8ex] 
  Длина & $\sigma$ \\ 
  \hline \\[-1.8ex] 
  $0$ & $51,045.670$ \\ 
  $1$ & $51,055.780$ \\ 
  $2$ & $51,097.950$ \\ 
  $3$ & $51,228.100$ \\ 
  $4$ & $51,135.910$ \\ 
  \hline \\[-1.8ex] 
  \end{tabular} 
  \end{table} 
\end{frame}

\begin{frame}[nologo]
  % Table created by stargazer v.5.2 by Marek Hlavac, Harvard University. E-mail: hlavac at fas.harvard.edu
  % Date and time: Tue, May 23, 2017 - 09:05:06 AM
  \begin{table}[!htbp] \centering 
    \caption{Параметры линейной регрессии для внешних измерений} 
    \label{} 
  \begin{tabular}{@{\extracolsep{5pt}}lc} 
  \\[-1.8ex]\hline 
  \hline \\[-1.8ex] 
   & \multicolumn{1}{c}{\textit{Зависимая переменная:}} \\ 
  \cline{2-2} 
  \\[-1.8ex] & $X$ \\ 
  \hline \\[-1.8ex] 
    $\beta_L$ & 78.012$^{**}$ ($-$15.081, 171.105) \\ 
    $\beta_0$ & 1.2e+6$^{***}$ (1,222,255, 1,222,711) \\ 
   \hline \\[-1.8ex] 
  $SE(\beta_L)$ & 51,112.740 (df = 999998) \\ 
  \hline 
  \hline \\[-1.8ex] 
  \textit{Примечание:}  & \multicolumn{1}{r}{$^{*}$p$<$0.1; $^{**}$p$<$0.05; $^{***}$p$<$0.01} \\ 
  \end{tabular} 
  \end{table} 
  $SE(\beta_L)$ -- стандартная ошибка для $\beta_L$
\end{frame}

\begin{frame}{Оценка требуемого числа измерений}
  Доверительный интервал для наклона в линейной модели: \\
  \begin{equation}
  \beta_L = \hat{\beta_L} \pm t^{*}_{df} \times SE(\beta_L)
  \end{equation}
  \begin{itemize}
   \item $t^{*}_{df}$ -- критическое значение t-критерия Стьюдента, определяется
         требуемым уроврем надёжности ($\alpha = 0.99$) и числом степеней свободы $df$\\
   \item $\hat{\beta_L}$ -- точечная оценка наклона\\
   \item $SE(\beta_L)$ -- стандартная ошибка $\beta_L$
  \end{itemize}
\end{frame}

\begin{frame}{Оценка требуемого числа измерений}
  Стандартная ошибка: \\
  \begin{equation}
  SE(\beta_L) = \frac{ \sigma_{res} }{ \sum (x_i - \bar{x})^2 }
  \end{equation}
  \begin{equation}
  \sigma_{res}^2 = \frac{ \sum (y_i - f(x_i))^2 }{ n - 2 }
  \end{equation}
  $\sigma_{res}$ -- несмещённая оценка дисперсии остатков
\end{frame}

\itmothankyou

\end{document}

