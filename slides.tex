\documentclass[14pt]{beamer}
\usepackage[english, russian]{babel}
\usepackage[utf8x]{inputenc}
\usepackage{itmobeamer}

\title[Выпускная Квалификационная Работа]{Атаки по сторонним каналам}
\author[]{Олейников Иван}
\institute[]{Университет ИТМО}
\date[]{Санкт-Петербург, 2017}

\begin{document}

\itmologoslide

\begin{darkbars}
    \begin{frame}[noheader,nologo,noframenumbering]
        \titlepage
    \end{frame}
\end{darkbars}

\begin{frame}[rulogoheader,nologo,noframenumbering]
    \itmoanothertitle
\end{frame}

\begin{frame}[englogoheader]{Заголовок раздела}
Что здесь должно быть?

Может быть план?
\end{frame}

\begin{frame}{Атаки по сторонним каналам}
    \begin{itemize}
        \item Атаки на кэш (access-based)
        \item Атаки по времени
        \item Атаки по энергопотреблению
        \item Электромагнитные атаки
        \item Акустические атаки
        \item Атаки со внесением ошибок
        \item Row-hammer
    \end{itemize}
\end{frame}

\begin{frame}{Ограничения атак по сторонним каналам}
    \begin{itemize}
        \item Требуется доступ к системе
            \begin{itemize}
                \item Возможность запуска программ на системе: атаки на кэш, Row-hammer
                \item Физический доступ: атаки по энергопотреблению, электромагнитные, акустические, внемением ошибок
            \end{itemize}
        \item Требуется высокоточное и дорогостоящее оборудование
            \begin{itemize}
                \item Самописцы, осциллографы: атаки по энергопотреблению, элкетромагнитные
                \item Микрофоны: акустические атаки
                \item Лазеры: атаки внесением ошибок
            \end{itemize}
    \end{itemize}
\end{frame}

\begin{frame}{Атаки по времени}
    \begin{itemize}
        \item Могут проводиться удалённо
        \item Намного проще других типов атак
        \item Современные процессоры позволяют измерять время с точностью до наносекунд
        \item Многие сетевые карты позволяют узнать точное время отправки/приёма пакета
        \item Не требуют наличия учёной степени по электротехнике
    \end{itemize}
\end{frame}

\begin{frame}{Цель работы}
    Исследовать насколько применима атака по времени на время работы программы под Linux.
~
    Задачи:
    \begin{itemize}
        \item Написать уязвимую к атаке по времени программу
        \item Снять измерения времени выполнения программы
        \item Проанализировать распределение полученных измерений
        \item Сделать выводы о применимости статистических методов к полученным измерениям
    \end{itemize}
\end{frame}

\begin{frame}{Модель атаки по времени}
    \begin{figure}
        \centering
        \includegraphics[height=0.9\textheight]{images/timing-attack-msc.png}
    \end{figure}
\end{frame}

\begin{frame}{Полученные распределения}
    \begin{figure}
        \centering
        \includegraphics[width=0.5\textwidth]{data/sca-playground.wiki/strcmp_Timing_Attack/attempt-2/plots/int-to-5-init-non-outliers.png}
        \includegraphics[width=0.5\textwidth]{data/sca-playground.wiki/strcmp_Timing_Attack/attempt-2/plots/ext-to-5-init.png}
        \caption{Q-Q plots}
    \end{figure}
\end{frame}

\begin{frame}{Kernel Density Estimate}
    \begin{figure}
        \centering
        \includegraphics[width=0.9\textwidth]{data/sca-playground.wiki/strcmp_Timing_Attack/attempt-2/plots/int-to-5-init-non-outliers-density.png}
        \caption{Внутренние измерения}
    \end{figure}
\end{frame}

\begin{frame}{Kernel Density Estimate}
    \begin{figure}
        \centering
        \includegraphics[width=0.9\textwidth]{data/sca-playground.wiki/strcmp_Timing_Attack/attempt-2/plots/ext-to-5-init-density.png}
        \caption{Внешние измерения}
    \end{figure}
\end{frame}

\begin{frame}{Параметры внутренней задержки}
  % Table created by stargazer v.5.2 by Marek Hlavac, Harvard University. E-mail: hlavac at fas.harvard.edu
  % Date and time: Tue, May 23, 2017 - 08:11:20 AM
  \begin{table}[!htbp] \centering 
    \caption{Точечные оценки среднеквадратичного отклонения} 
    \label{} 
  \begin{tabular}{@{\extracolsep{5pt}} ccc} 
  \\[-1.8ex]\hline 
  \hline \\[-1.8ex] 
   & length & cycles \\ 
  \hline \\[-1.8ex] 
  1 & $0$ & $51.764$ \\ 
  2 & $1$ & $49.222$ \\ 
  3 & $2$ & $50.634$ \\ 
  4 & $3$ & $50.772$ \\ 
  5 & $4$ & $52.566$ \\ 
  \hline \\[-1.8ex] 
  \end{tabular} 
  \end{table} 
\end{frame}

\begin{frame}{Параметры внутренней задержки}
% Table created by stargazer v.5.2 by Marek Hlavac, Harvard University. E-mail: hlavac at fas.harvard.edu
% Date and time: Tue, May 23, 2017 - 08:03:32 AM
\begin{table}[!htbp] \centering 
  \caption{Доверительные интервалы мат. ожидания} 
  \label{} 
\begin{tabular}{@{\extracolsep{5pt}} cccc} 
\\[-1.8ex]\hline 
\hline \\[-1.8ex] 
 & length & left & right \\ 
\hline \\[-1.8ex] 
1 & $0$ & $879.413$ & $879.867$ \\ 
2 & $1$ & $892.966$ & $893.398$ \\ 
3 & $2$ & $927.871$ & $928.315$ \\ 
4 & $3$ & $958.985$ & $959.430$ \\ 
5 & $4$ & $1,009.111$ & $1,009.571$ \\ 
\hline \\[-1.8ex] 
\end{tabular} 
\end{table} 
\end{frame}

\begin{frame}
% Table created by stargazer v.5.2 by Marek Hlavac, Harvard University. E-mail: hlavac at fas.harvard.edu
% Date and time: Tue, May 23, 2017 - 09:01:52 AM
\begin{table}[!htbp] \centering 
  \caption{Линейная модель для внутренних измерений} 
  \label{} 
\begin{tabular}{@{\extracolsep{5pt}}lc} 
\\[-1.8ex]\hline 
\hline \\[-1.8ex] 
 & \multicolumn{1}{c}{\textit{Dependent variable:}} \\ 
\cline{2-2} 
\\[-1.8ex] & cycles \\ 
\hline \\[-1.8ex] 
 length & 32.556$^{***}$ (32.461, 32.650) \\ 
  Constant & 868.782$^{***}$ (868.551, 869.013) \\ 
 \hline \\[-1.8ex] 
Adjusted R$^{2}$ & 0.442 \\ 
Residual Std. Error & 51.739 (df = 999702) \\ 
F Statistic & 791,637.300$^{***}$ (df = 1; 999702) \\ 
\hline 
\hline
\textit{Note:}  & \multicolumn{1}{r}{$^{*}$p$<$0.1; $^{**}$p$<$0.05; $^{***}$p$<$0.01} \\ 
\end{tabular} 
\end{table} 
\end{frame}

\begin{frame}{Параметры внешней задержки}
% Table created by stargazer v.5.2 by Marek Hlavac, Harvard University. E-mail: hlavac at fas.harvard.edu
% Date and time: Tue, May 23, 2017 - 08:29:22 AM
\begin{table}[!htbp] \centering 
  \caption{Точечные оценки среднеквадратичного отклонения} 
  \label{} 
\begin{tabular}{@{\extracolsep{5pt}} ccc} 
\\[-1.8ex]\hline 
\hline \\[-1.8ex] 
 & length & cycles \\ 
\hline \\[-1.8ex] 
1 & $0$ & $51,045.670$ \\ 
2 & $1$ & $51,055.780$ \\ 
3 & $2$ & $51,097.950$ \\ 
4 & $3$ & $51,228.100$ \\ 
5 & $4$ & $51,135.910$ \\ 
\hline \\[-1.8ex] 
\end{tabular} 
\end{table} 
\end{frame}

\begin{frame}
% Table created by stargazer v.5.2 by Marek Hlavac, Harvard University. E-mail: hlavac at fas.harvard.edu
% Date and time: Tue, May 23, 2017 - 09:05:06 AM
\begin{table}[!htbp] \centering 
  \caption{Линейная модель для внешних измерений} 
  \label{} 
\begin{tabular}{@{\extracolsep{5pt}}lc} 
\\[-1.8ex]\hline 
\hline \\[-1.8ex] 
 & \multicolumn{1}{c}{\textit{Dependent variable:}} \\ 
\cline{2-2} 
\\[-1.8ex] & cycles \\ 
\hline \\[-1.8ex] 
 length & 78.012$^{**}$ ($-$15.081, 171.105) \\ 
  Constant & 1.2e+6$^{***}$ (1,222,255, 1,222,711) \\ 
 \hline \\[-1.8ex] 
Adjusted R$^{2}$ & 0.00000 \\ 
Residual Std. Error & 51,112.740 (df = 999998) \\ 
F Statistic & 4.659$^{**}$ (df = 1; 999998) \\ 
\hline 
\hline \\[-1.8ex] 
\textit{Note:}  & \multicolumn{1}{r}{$^{*}$p$<$0.1; $^{**}$p$<$0.05; $^{***}$p$<$0.01} \\ 
\end{tabular} 
\end{table} 
\end{frame}

\begin{frame}{Выводы}

\begin{equation}
X = \bar{X} \pm Z \frac{ \sigma }{ \sqrt{n} }
\end{equation}

На уравнении представлена формула доверительного интервала для мат. ожидания.
Для уменьшения предела погрешности в $C$ раз, не изменяя доверительную вероятность $\alpha$, от которой зависит $Z$,
потребуется увеличить $n$ в $C^2$ раз.

Линейная модель для внешних измерений с $\alpha = 0.99$ и $n =10^6$ имеет предел погрешности равный $186$ циклов.
Для атаки по времени на программу сравнивающую строки, необходим предел погрешности равный $3-4$ такта. Для его достижения
нужно увеличить $n$ в $( \frac{186}{4} )^2 = 162 $ раза. Для снятия такого числа измерения потребуется
$162 / 24 \approx 40$ суток.

\end{frame}

\begin{frame}[nologo]
\end{frame}

\itmothankyou

\end{document}

