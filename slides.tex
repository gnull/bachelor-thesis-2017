\documentclass[14pt]{beamer}
\usepackage[english, russian]{babel}
\usepackage[utf8x]{inputenc}
\usepackage{itmobeamer}

\title[Выпускная Квалификационная Работа]{Атаки по сторонним каналам}
\author[]{Олейников Иван}
\institute[]{Университет ИТМО}
\date[]{Санкт-Петербург, 2017}

\begin{document}

\itmologoslide

\begin{darkbars}
    \begin{frame}[noheader,nologo,noframenumbering]
        \titlepage
    \end{frame}
\end{darkbars}

\begin{frame}[rulogoheader,nologo,noframenumbering]
    \itmoanothertitle
\end{frame}

\begin{frame}[englogoheader]{Заголовок раздела}
Что здесь должно быть?

Может быть план?
\end{frame}

\begin{frame}{Атаки по сторонним каналам}
    \begin{itemize}
        \item Атаки на кэш (access-based)
        \item Атаки по времени
        \item Атаки по энергопотреблению
        \item Электромагнитные атаки
        \item Акустические атаки
        \item Атаки со внесением ошибок
        \item Row-hammer
    \end{itemize}
\end{frame}

\begin{frame}{Ограничения атак по сторонним каналам}
    \begin{itemize}
        \item Требуется доступ к системе
            \begin{itemize}
                \item Возможность запуска программ на системе: атаки на кэш, Row-hammer
                \item Физический доступ: атаки по энергопотреблению, электромагнитные, акустические, внемением ошибок
            \end{itemize}
        \item Требуется высокоточное и дорогостоящее оборудование
            \begin{itemize}
                \item Самописцы, осциллографы: атаки по энергопотреблению, элкетромагнитные
                \item Микрофоны: акустические атаки
                \item Лазеры: атаки внесением ошибок
            \end{itemize}
    \end{itemize}
\end{frame}

\begin{frame}{Атаки по времени}
    \begin{itemize}
        \item Могут проводиться удалённо
        \item Намного проще других типов атак
        \item Современные процессоры позволяют измерять время с точностью до наносекунд
        \item Многие сетевые карты позволяют узнать точное время отправки/приёма пакета
        \item Не требуют наличия учёной степени по электротехнике
    \end{itemize}
\end{frame}

\begin{frame}{Цель работы}
    Исследовать, насколько применима атака по времени на время работы программы под Linux.
~
    Задачи:
    \begin{itemize}
        \item Написать уязвимую к атаке по времени программу
        \item Снять измерения времени выполнения программы
        \item Проанализировать распределение полученных измерений
        \item Сделать выводы о применимости статистических методов к полученным измерениям
    \end{itemize}
\end{frame}

\begin{frame}{Модель атаки по времени}
    \begin{figure}[h]
        \centering
        \includegraphics[height=5cm]{images/timing-attack-msc.png}
    \end{figure}
\end{frame}

\begin{frame}[Полученные распределения]
    \begin{figure}
        \centering
        \includegraphics[width=0.45\textwidth]{data/sca-playground.wiki/strcmp\_Timing\_Attack/attempt-2/plots/int-to-5-init-non-outliers.png}
        \includegraphics[width=0.45\textwidth]{data/sca-playground.wiki/strcmp\_Timing\_Attack/attempt-2/plots/ext-to-5-init.png}
    \end{figure}
\end{frame}

% \begin{frame}{Kernel Density Estimate}
%     \begin{figure}
%         \centering
%         \includegraphics[width=0.9\textwidth]{data/sca-playground.wiki/strcmp_Timing_Attack/attempt-2/plots/int-to-5-init-non-outliers-density.png}
%     \end{figure}
% \end{frame}

% \begin{frame}{Kernel Density Estimate}
%     \begin{figure}
%         \centering
%         \includegraphics[width=0.9\textwidth]{data/sca-playground.wiki/strcmp_Timing_Attack/attempt-2/plots/ext-to-5-init-non-outliers-density.png}
%     \end{figure}
% \end{frame}

\begin{frame}{Заголовок}
    \begin{itemize}
        \item Первый уровень списка
            \begin{itemize}
                \item Второй уровень списка
                    \begin{itemize}
                        \item Третий уровень списка\\
                            ~
                    \end{itemize}
            \end{itemize}
        \item Первый уровень списка\\
            ~
        \item Первый уровень списка\\
            ~
    \end{itemize}
\end{frame}

\begin{frame}{Заголовок}
    \begin{itemize}
        \item Первый уровень списка
            \begin{itemize}
                \item Второй уровень списка
                    \begin{itemize}
                        \item Третий уровень списка\\
                            ~
                    \end{itemize}
            \end{itemize}
        \item Первый уровень списка\\
            ~
        \item Первый уровень списка\\
            ~
    \end{itemize}
\end{frame}

\begin{frame}[nologo]
\end{frame}

\itmothankyou

\end{document}

