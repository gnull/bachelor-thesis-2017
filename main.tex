\documentclass[11pt,twoside,a4paper]{report}

% Copy-paste from https://en.wikibooks.org/wiki/LaTeX/Internationalization#Cyrillic_script
\usepackage[T1,T2A]{fontenc}
\usepackage[utf8]{inputenc}
\usepackage[russian]{babel}

\begin{document}

\title{Выпускная Квалификационная Работа. Атаки по сторонним каналам}
\author{Олейников Иван}
\date{\today}
\maketitle

\chapter*{Задание на ВКР}

TODO: Выянить, что это такое.

\chapter*{Аннотация}

TODO: Это разве не часть поянительной записки?

\chapter*{Список обозначений}

\chapter*{Термины}

\chapter*{Введение}

Информация. Безопасность. Будущее. Технологии.

Описание содержимого глав.

\chapter{Атаки по времени}

Что такое атаки по сторонним каналам. Что такое атаки по времени, чем они примечательны.

Разные сценарии атак по времени.

Методы анализа данных при атаках по времени. Статистические методы.

\chapter{Опериционная система Linux}

Ядро линукс. Выполнение системного вызова execve.

Приём пакета UDP-сокетом.

\chapter{Эксперименты}

\section{Используемые инструменты}

\section{Методика провдения экспериментов}

\section{Анализ результатов}

\end{document}

