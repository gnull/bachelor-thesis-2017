TODO: Введение описано не достаточно полно. Нужно больше текста. \\
\\

Классическая (или теоретическая?) криптография рассматривает сценарии, в
которых атакующий получает доступ к некоторым параметрам и/или результатам
работы криптографических алгоритмов. При этом, не рассматриваются детали
конкретной программной или аппаратной реализации алгоритма, внимание уделяется
только математическому описанию алгоритма.

На практике же математические описания криптографических алгоритмов реализуются
в виде программы или цифровой схемы. Эти реализации могут предоставлять атакующему
много дополнительной информации о работе алгоритма. Такой информацией может быть
электромагнитное излучение, исходящее от шифрующего устройства, потрябляемая им
мощность, издаваемые им звуки, затраченное на вычисления время и много других
параметров.

Атаки по времени интересны тем, что они допускают сценарии, при которых не
требуется ни активного вмешательства в
работу шифрующего устройства ни физического доступа к нему для снятия изменений.
Атакующий может находиться в одной сети с шифрующим устройством и измерять
время, которое уходит у него на обработку запросов на шифрование, которые
поступают от легитимных клиентов. Кроме того, он может удалённо слать такие
запросы по сети или обращаться с ними к шифрующему процессу, имея доступ
непривелигированного пользователя к шифрующей машине.

Основным препятствием для проведения успешной атаки по времени является шум,
который присутствует в измерениях, снимаемых атакующим. Этот шум вызывается
множеством случайных процессов, которые влияют на измеряемое атакующим время.
